\section{Codul sursă}
\subsection{\texttt{main.cpp}}
{
\hbadness=10000
Acesta este fișierul care conține funcția principală a programului. Nu sunt multe modificări aici.
\inputminted[
 tabsize=2,
 breaklines,
 frame=lines,
 framesep=2mm,
 baselinestretch=1.2,
 fontsize=\footnotesize,
 linenos
]{c}{../main.cpp}


\subsection{\texttt{calculator.h}}
Acesta este fișierul header, în care sunt declarate clasele, precum și funcțiile care vor urma să fie declarate.
\inputminted[
 tabsize=2,
 breaklines,
 frame=lines,
 framesep=2mm,
 baselinestretch=1.2,
 fontsize=\footnotesize,
 linenos
]{c}{../calculator.h}

\subsection{\texttt{calculator.cpp}}
Acesta este fișierul cel mai important din proiect. Este fișierul care declară funcțiile necesare, creează variabilele și spune programului ce funcții trebuie folosite la apăsarea căror butoane.
\inputminted[
 tabsize=2,
 breaklines,
 frame=lines,
 framesep=2mm,
 baselinestretch=1.2,
 fontsize=\footnotesize,
 linenos
]{c}{../calculator.cpp}

\subsection{\texttt{calculator.ui}}
Acesta este fișierul care conține layout-ul grafic al programului (ex.: unde se află butoanele, felul în care arată etc.). Acesta a fost auto-generat de către componenta de `Design` din interiorul IDE-ului Qt Creator (secțiunea \texttt{\ref{qtcreator}}). Este, de asemenea, și cel mai lung fișier din întregul proiect.
\inputminted[
 tabsize=2,
 breaklines,
 frame=lines,
 framesep=2mm,
 baselinestretch=1.2,
 fontsize=\footnotesize,
 linenos
]{c}{../calculator.ui}
}
