\section{Instrucțiuni pentru compilare}
Personal, am compilat acest program doar pe Linux, așa că instrucțiunile mele vor fi puțin neclare pentru alte sisteme de operare.

Prima dată, trebuie să instalați Qt Creator:

\subsection{Pentru Linux}
Pe Linux, tot ce trebuie să faceți este să instalați Qt Creator cu ajutorul managerul-ui de pachete al distribuției dumneavoastră, sau cu ajutorul Flatpak:
\begin{itemize}
 \item bazate pe Debian (ex: Ubuntu, Linux Mint, Elementary OS, Pop\_OS etc.):\\
  \texttt{\$ sudo apt install qtcreator}
 \item bazate pe RHEL (ex: Fedora, CentOS, Rocky Linux etc.):\\
  \texttt{\$ sudo dnf install qt-creator}
 \item bazate pe Arch (ex: Manjaro, EndeavourOS, Arco Linux etc.):\\
  \texttt{\$ sudo pacman -Syu qtcreator}
 \item cu Flatpak (nu contează distribuția):\\
  \texttt{\$ sudo flatpak install io.qt.QtCreator}
\end{itemize}

\subsection{Pentru Windows/macOS/altele}
Pentru restul sistemelor de operare, sau în cazul (extrem de rar) în care nu puteți găsi pachetul pentru distribuția de Linux pe care o folosiți, puteți intra pe linkul \url{www.qt.io/download-qt-installer} și să descărcați installer-ul de acolo. Dar, din încercările mele, am observat că acel instalator necesită un cont pe site-ul Qt.

\vspace*{1cm}

După ce ați instalat Qt Creator, puteți deschide direct proiectul inclus cu CD-ul care vine cu documentația, sau să copiați codul din acest document într-un proiect nou.
