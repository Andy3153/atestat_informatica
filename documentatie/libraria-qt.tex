\section{Librăria Qt}
\subsection{Introducere}
Qt este un framework cross-platform, creat cu scopul de a fi o librărie utilizată pentru crearea interfețelor grafice (GUI\footnote{engleză: graphical user interface (interfață grafică)}), precum și pentru crearea de programe care rulează pe diferite platforme software ca și Linux, Windows, macOS, Android, iOS sau sisteme embedded, fără a fi necesare schimbări în cod, de cele mai multe ori, în timp ce programul scris este un program nativ platformei, cu capabilități și viteză native.

Qt este dezvoltat de Compania Qt și de Proiectul Qt, sub guvernanță open-source. Qt este valabil sub o licență comercială, dar și sub o licență open-source (GPL și LGPL)\footnote{GNU General Public License și GNU Lesser General Public License: \url{www.gnu.org/licenses}}.

Cele mai cunoscute utilizări ale Qt sunt mediul de desktop KDE Plasma pentru sistemele de operare Unix-like\footnote{\url{en.wikipedia.org/wiki/Unix-like}} (ex.: Linux/BSD), precum și programele VirtualBox, Google Earth, Autodesk Maya, Autodesk 3ds Max, DaVinci Resolve, OBS, VLC, Wireshark, qBittorrent.

\subsection{Qt Creator} \label{qtcreator}
Qt Creator este un IDE\footnote{engleză: integrated development environment (mediu integrat de dezvoltare)} cross-platform pentru C++, JavaScript și QML care simplifică crearea programelor cu o interfață grafică.
Este parte din SDK\footnote{engleză: software development kit (trusă de dezvoltare a programelor)}-ul pentru framework-ul Qt pentru crearea aplicațiilor GUI. Include un debugger vizual și un designer de interfețe grafice WYSIWYG\footnote{engleză: what you see is what you get (ceea ce vezi este ceea ce primești)}

Editorul acestuia include caracteristici comune în alte IDE-uri, precum evidențierea sintaxei prin culori, auto-completare și mesaje despre erori în cod de la servere LSP.

Qt Creator poate folosi compilatorul GNU GCC, MinGW, Visual C++ sau Clang.

\subsection{De ce am ales Qt?}
Motivele pentru care am ales să mă folosesc de librăria Qt pentru a-mi scrie proiectul de atestat sunt următoarele:
\begin{itemize}
 \item folosesc zilnic, de cel puțin trei ani, unul din cele mai cunoscute proiecte în care se utilizează Qt, mediul de desktop KDE Plasma și suita de programe KDE, lucru care m-a convins că este un framework robust și de încredere
 \item am vrut să învăț ceva nou
 \item am considerat că învătarea folosirii unui framework pentru crearea programelor cu interfață grafică al cărui proiecte se pot compila, mai târziu, pe aproape toate platformele software importante în ziua de astăzi este ceva care va îmi va fi util în viitor
\end{itemize}
