\section{Limbajul de programare C++}
\subsection{Introducere}
C++ este un limbaj de programare general creat de către informaticianul Bjarne Stroustrup de la Bell Labs în anul 1985 ca o extensie a limbajului de programare C, numit inițial „C cu clase”.

Limbajul s-a maturizat destul de mult în timp, iar C++ are acum programare orientată pe obiecte și caracteristici mai avansate, precum manipulare low-level a memoriei.

Este un limbaj de programare compilat și există multe compilatoare create de diferite companii/organizații, precum Fundația Software-ului Liber (GNU GCC), LLVM (Clang), Microsoft (Visual C++), Intel (icx), Oracle (Oracle Developer Studio C++) și IBM (IBM XL C++).

C++ este standardizat de ISO\footnote{engleză: International Organization for Standardization (Organizația Internațională de Standardizare)}, cea mai recentă versiune a standardului fiind publicată în decembrie 2020, numită C++20.

C++ a fost dezvoltat în anii 1980, ca o serie de îmbunătățiri ale limbajului C. Acestea au început cu adăugarea noțiunii de clase, apoi de funcții virtuale, suprascrierea operatorilor, moștenire multiplă, șabloane și excepții. Limbajul a fost standardizat în anul 1998 ca și ISO 14882:1998\footnote{\url{www.iso.org/standard/25845.html}}.

În anii 1990, C++ a devenit unul dintre cele mai populare limbaje de programare comerciale, rămânând astfel până în ziua de astăzi.

\subsection{De ce am ales C++?}
Motivele pentru care am ales să îmi scriu proiectul de atestat în limbajul C++ sunt următoarele:
\begin{itemize}
 \item am experiență anterioară cu acesta
 \item este un limbaj de programare folosit în multe domenii
 \item chiar dacă librăria grafică în care am scris programul este valabilă și pentru alte limbaje de programare, precum Python, C\# sau Rust, Qt are cel mai mult suport pentru C++
\end{itemize}
